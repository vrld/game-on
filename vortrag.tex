\documentclass{beamer}
%\usepackage{ngerman}
\usepackage{inputenc}
\usepackage{listings}
\usepackage[T1]{fontenc}

\definecolor{hilight}{rgb}{.1,.3,.4}
\definecolor{lolight}{rgb}{.7,.7,.7}
\definecolor{url}{rgb}{.2,.5,.7}

\setbeamercovered{transparent=0}
\newcommand{\sframe}[1]{\frame{\huge \begin{center} #1 \end{center}}}
\newcommand{\tframe}[2]{\frame{\frametitle{#1}#2}}
\newcommand{\cframe}[2]{\frame{\frametitle{#1}\large \begin{center} #2 \end{center}}}
\newcommand{\hilight}[1]{\textbf{\color{hilight}{#1}}}
\newcommand{\lolight}[1]{\textit{\color{lolight}{#1}}}
\newcommand{\code}[1]{\texttt{#1}}
\renewcommand{\url}[1]{\texttt{\color{url}{#1}}}

\newcommand{\sprod}[2]{\langle #1, #2 \rangle}
\DeclareMathOperator*{\argmax}{arg\,max}
\beamertemplatenavigationsymbolsempty

\title{Game On}
\subtitle{Spiele programmieren mit L�VE}
\author{vrld}
\date{GPN 11}

\mode<presentation>{
	\useoutertheme{shortinfoline}
	\useinnertheme{rounded}
}

\begin{document}

\sframe{
	\Huge \hilight{Game On} \\[.4cm]
	\normalsize
	Spiele programieren mit L�VE \\[.3cm]
	\lolight{GPN 11 // vrld}
}

\begin{frame}[fragile]
	\frametitle{Hello, L�VE}
	\begin{lstlisting}
		function love.load()
		    love.graphics.print("Hello, L�VE", 10, 10)
		end
	\end{lstlisting}
\end{frame}

\sframe{
	\includegraphics[width=6cm]{lua-logo-nolabel.png}
}

\tframe{Lua}{
	\begin{itemize}
		\item KISS Prinzip
		\item Tables als universelle Datenstruktur
		\item Prozedual, Objekt-Orientier, Funktional, \(\dots\)
		\item Programming in Lua: \url{www.lua.org/pil}
		\item Referenz: \url{www.lua.org/manual/5.1}
		\item Lua Users: \url{lua-users.org/wiki}
		\item Lua Missions: \url{github.com/kikito/lua\_missions}
	\end{itemize}
}

\sframe{
	\includegraphics[width=6cm]{love.png}\\
	\small (Baby don't hurt me)
}

\tframe{L�VE}{
	\begin{itemize}
		\item Open Source 2D Game Framework
		\item Lua Scripting
		\item Windows, OSX, Linux (+ Dingoo, Caanoo, OpenPandora)
		\item \url{www.love2d.org}
		\item \url{www.love2d.org/wiki} - Einstiegshilfe
		\item \url{www.love2d.org/forums} - Ausf�hrliche Hilfe
		\item \url{\#love@irc.freenode.net} - Schnelle Hilfe
	\end{itemize}
}

\tframe{Running stuff}{
	Running stuff:
	\begin{itemize}
		\item Drag and Drop
		\item \code{love /path/to/game/folder}
		\item \code{love /path/to/game.love}
	\end{itemize}
	\vspace{.3cm}

	.love:
	\begin{itemize}
		\item Umbenannte .zip Datei
		\item \code{main.lua} im top-level
		\item Alles andere egal
	\end{itemize}
}

\cframe{Gameloop}{
	\includegraphics[width=10cm]{gameloop.png}
}

\begin{frame}[fragile]
	\frametitle{Hamsterball}
	\small
	\begin{lstlisting}
function love.load()
    hamster = {
        img = love.graphics.newImage('hamster.png'),
        x = 400, y = 300
    }
end

function love.update(dt)
    if love.keyboard.isDown('up') then
        hamster.y = hamster.y - dt * 50
    elseif love.keyboard.isDown('down') then
        hamster.y = hamster.y + dt * 50
    end
    ...
end

function love.draw()
    love.graphics.draw(hamster.img, hamster.x, hamster.y)
end
	\end{lstlisting}
\end{frame}

\cframe{Libraries}{
	\begin{itemize}
		\item \hilight{An} \hilight{A}nimation \hilight{L}ibrary \textit{(von bartbes)}
			\begin{itemize}
				\item Sprite-Animationen
				\item \url{love2d.org/wiki/AnAL}
			\end{itemize}
		\item \hilight{S}imple \hilight{E}ducative \hilight{C}lass \hilight{S}ystem \textit{(von bartbes)}
			\begin{itemize}
				\item Einfaches Klassen-system
				\item \url{love2d.org/wiki/SECS}
			\end{itemize}
		\item MiddleClass und MiddleClass Extras \textit{(von kikito)}
			\begin{itemize}
				\item Klassen-system mit vielen Features
				\item \url{github.com/kikito/MiddleClass}
			\end{itemize}
		\item \hilight{H}elper \hilight{U}tilities for \hilight{M}assive \hilight{P}rogress \textit{(von mir)}
			\begin{itemize}
				\item Gamestates, Timer, Vektoren, Klassen-System, Kamera, Ringbuffer
				\item \url{vrld.github.com/hump}
			\end{itemize}
		\item HardonCollider \textit{(von mir)}
			\begin{itemize}
				\item Kollisionserkennung
				\item \url{vrld.github.com/HardonCollider}
			\end{itemize}
	\end{itemize}
}

\end{document}
